\documentclass[11pt]{article}
\def\convw #1{\mathop{\rightarrow}\limits^{#1}}
\usepackage{fullpage}
\title{A Simple Introduction to \LaTeX}
\author{Merlise Clyde}
\date{September 10, 2003}


\begin{document}
\maketitle

\section*{Getting Started}
A basic \LaTeX file has a two major sections: the  preamble to set the
style options, define variables and new commands, followed by  the guts
of the document, as shown below in the file {\tt latex-graphics.tex}:
\begin{verbatim}
\documentclass[12pt]{article}
\usepackage{graphicx, fullpage}
\title{Intoduction to Graphics in \LaTeX}
\author{Merlise Clyde}
\date{September 10, 2003}

% end of preamble
   
\begin{document}
\maketitle
Here is my first document! 
  
\end{document}
\end{verbatim}
The first line indicates that the document will be in a 12 point
font and follows the {\tt article} style file.  There are other options
for style files, including seminar (for slides), book and thesis styles.

The \verb!\usepackage! command is used to add additional packages, in
this case a package for including graphics and a package for formatting
pages with one inch margins.  There are lots of other packages!  The
\verb!\author!, \verb!\title!, and \verb!\date!  commands are included
in the preamble to define the author, title, and date and are used in
making a title for the document, using the {\tt \\maketitle} command.
You can omit the date, and by default the maketitle command will use
today's date.  The \verb!\maketitle! command is also useful in conjunction with
\verb!\begin{titlepage}! and \verb!\begin{abstract}!, useful for
    technical reports.  These are optional in the article environment.
    Comments can be included anywhere after \%.  In case you are
    wondering about any typesetting commands used to produce this
    handout, this file {\tt latex-intro.tex} is also available on the
    web page, and you can peruse it at your leisure.

\section*{Creating a DVI File}
Once you have created a \LaTeX file, there are two ways to ``Compile'' a
LaTeX file to create a DeVice Independent (dvi) file, which can be
previewed.  The first that will work on both ISDS and ACPUB machines
involves running LaTeX from a UNIX terminal window: 
\begin{verbatim}
> latex latex-graphics
\end{verbatim}
Note, it is not necessary to specify the {\tt .tex} file ending. 

The second way  is to use a special LaTeX mode in emacs called {\tt AUCTeX}.
In  emacs, you should see  ``LaTeX'' on
the info bar near the bottom of the screen when you open a file ending
in {\tt .tex}.   This mode will also highlight functions and text in
different colors, which can be useful in debugging a tex file with
errors.

In this mode, you can use {\tt C-c C-c}
to access a set of useful  \LaTeX commands.  For example, {\tt C-c C-c
  latex} will run \LaTeX\ on the current file.  If it compiles
successfully and completely (sometimes you need to run it more than
once if there are internal references), then {\tt C-c C-c View} will
bring up an xdvi window so that you can preview the dvi file.  Note that
Emacs is pretty good at guessing what you want to do when you hit {\tt
  C-c C-c}, so if the default is correct, you can just hit enter.  Other
useful commands are {\tt   Print} and {\tt Spell}.  For a complete list
of choices, you can hit {\tt C-c C-c} and hit the tab key.  Note that
you can use tab-completions for the options.  

If you are not using emacs, then you can preview the dvi file by
entering  after the unix prompt
\begin{verbatim}
> xdvi latex-graphics &
\end{verbatim}
in a unix terminal window.  Again, the dvi file ending is not
necessary.  Keep this window open; as you add to the \LaTeX file and
re-compile, the dvi window will be updated.  You may need
to click on the Reread button to reload the file and update the window,
if the changes do not automatically appear.

\section*{Printing}
There are two options for creating a file that can be printed.  
\subsection*{Creating a Postscript File}
The first option uses the unix command 
{\tt dvips}  to create a
postscript ({\tt .ps}) file from the {\tt .dvi} file.  On some
systems, it will send this file directly to the printer, and not save
anything to disk unless you explicitly tell it to with the \verb!-o!
switch.   In ISDS if
you use {\tt dvips} from the unix prompt, it will only create a postscript
file, and not send anything to the printer.  You can then print by
using {\tt lpr -Pmyprinter myfile.ps}, where -Pmyprinter specifies
sending the output to the printer named myprinter.

You can print two pages in smaller print one one piece of paper
(sideways) with the {\tt psnup} command.  One way to do this at the
unix prompt is to run dvips and then {\it pipe} the output to the psnup
function and then pipe that output to the lpr function:
\begin{verbatim}
dvips latex-example -f | psnup -n 2 | lpr -P211
\end{verbatim}
where the printer name here is {\tt 211}.  

Note that {\tt dvips} has many options.  Important ones include those
that specify a print range, such as \verb!-p!, \verb!-n!, and
\verb!-f!.  For more info, see the {\tt man} page for {\tt dvips},
\begin{verbatim} 
man dvips
\end{verbatim}

\subsection*{Creating a PDF File}
While postscript is meant to be portable across many
environments/printers, in reality this is not always the case.  The
Adobe Portable Document Format is meant to be a better solution (but
doesn't always succeed either).  To create a PDF file, there are a
couple of options.  The easiest way is to use the command
\begin{verbatim}
> pdflatex latex-graphics
\end{verbatim}
to create a PDF file.  This option will require that all graphs in any
figure are in a PDF format or non-postscript format as described later.

\section*{Other Parts of the Document}
All of the commands below should be contained within the document
environment between the 
\begin{verbatim}
\begin{document}
 
\section{Introduction} 
Let's get started!
\subsection{How to add sections}
use the command....
\section{Adding Figures}

etc

\end{document}
\end{verbatim}

\subsection*{Sections}
To create section and subsection headings use the command 
\begin{verbatim}
\section{EDA}
\subsection{Quantile-Quantile Plots}
\end{verbatim}
You will note that all the sections and subsections in this 
document are not numbered. If you do not
want numbered sections, you can instead use \verb!\section*! and 
\verb!\subsection*!.  

\subsection*{Comments}
Comments are started by the percent sign (\%).  Anything that appears
on a line after a \% is ignored by \LaTeX.  This can be helpful when
making changes to a document.  But this is why you must always
remember to put a backslash before a percent sign when you want a
percent sign in your text, i.e., \verb!\%!.  Try removing the
backslash in the example file before the percent sign of 
\verb!4\% margin of error!, recompile, and see what happens.  To comment
out a large block of text include it in a comment environment by using
\begin{verbatim}
\begin{comment}
text to comment out
\end{comment}
\end{verbatim}
To use this environment, add {\tt comment} to the list of packages in
the \verb!\usepackage! command.

\subsection*{Equations and Mathematics}
Within the text you may want to include formulae or Greek symbols. To do
this, you need to invoke math mode with a 
single dollar sign and type in the expression, for example for a
fraction 1/2 use \verb!$\frac{1}{2}$! command, which
takes two arguments, the numerator and the denominator.  For an
expression using Greek letters, ${\rm E}[Y_i] = \beta_0 + \beta_1 X_i$, use
\verb!${\rm E}[Y_i] = \beta_0 + \beta_1 X_i$!
If in doubt about how to get the right text for ann expression, use the
\LaTeX menu in emacs to invoke Math Mode (under Miscellaneous) or use
{\tt C-c ~}, and of course look it up in your favorite \LaTeX book.

For a more complicated expression you may want to offset it from the
text using a displayed equation environment.

\begin{equation}
p(x_i|\theta) = \theta^{x_i}(1-\theta)^{1-x_i} = \left\{ \begin{array}{ll}
                                           \theta& \mbox{if $x_i=1$,} \\ 
                       1-\theta& \mbox{if $x_i=0$.} \end{array} \right.
\end{equation}
by using
\begin{verbatim}
\begin{equation}
p(x_i|\theta) = \theta^{x_i}(1-\theta)^{1-x_i} = \left\{ \begin{array}{ll}
                                           \theta& \mbox{if $x_i=1$,} \\ 
                       1-\theta& \mbox{if $x_i=0$.} \end{array} \right.
\end{equation}
\end{verbatim}
To get a curly brace to print, you must use \verb!\{!, since the curly
brace is used within \LaTeX\ to bound action areas.  You can mix and match your
\verb!\left! and \verb!\right!, as long as you have one of each.  You
can also use \verb!\left.! to match any \verb!\right! but to not print
anything (since they must be balanced). 
Note that an {\tt array} is just like a {\tt tabular}, except that it
must be used in math mode, and everything inside is automatically in
math mode.  Also note that \verb!\mbox{}! is a method of including
regular text (not in math mode) inside of math mode.  Try the above
expression so you can see all of the parts in action.

With the equation environment, equations are automatically numbered
throughout the document. To 
skip numbers, you  could equally use \verb!$$! and \verb!$$!\verb!\[! and \verb!\]!, or
\verb!\begin{equation*}! and \verb!\end{equation*}!.   This is analogous to the difference between
\verb!\begin{section}! and \verb!\begin{section*}!.
You can get \LaTeX\  to make an automatic reference with the
\verb!\label{eq:bernoulli}! and \verb!\ref{eq:bernoulli}! commands.

\section*{Figures}
To include a figure in a document, use the figure and centering environments.
\begin{verbatim}
\begin{figure}[htb]
\begin{center}
\includegraphics[height=1in,width=1in,angle=-90]{foo}
\caption{This is a figure.}
\end{center}
\end{figure}

\end{verbatim}

The syntax of the \verb!\includegraphics! command is as follows:

\begin{verbatim}
\includegraphics[parameters]{filename}
\end{verbatim}

where parameters is a comma-separated list of any of the
following: 
\begin{verbatim}
           bb=llx lly urx ury (bounding box),
           width=h_length, height=v_length,
           angle= angle,  scale=factor,
           clip=true/false, draft=true/false.
\end{verbatim}
Use the \verb!\includegraphics! command with the name of the
graphic file {\tt foo} without extension.  By doing so
each of {\bf latex} and {\bf pdflatex} will choose the correct
version of the file to be included, i.e., 
foo.ps and foo.pdf (or
foo.png, or whatever graphic format you have chosen).

\section*{Other Environments}
The AUCTeX mode for LaTeX has several built in function to insert LaTeX
commands.  You can explore options under the LaTeX menu.  Most of these
are ``bound'' to particular Control or Meta key sequences when in LaTeX
mode. 

\section*{Finishing Up}
When you are done, make sure to close any {\tt xdvi} windows and exit
emacs before logging off.  Failure to do so could spawn runaway
processes that gobble up memory!


\end{document}




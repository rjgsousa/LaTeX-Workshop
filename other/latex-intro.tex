
%% Revised 06/22/02 (cmihelic)
%% 2002 TA Revisions 06/22/02 (sasen)
%% Further revised 06/25/02; slides numbered (cmihelic)
%% Revised 2003 TA: 06/21/03 (bveto)
%% More revisions: 06/24/03 (sasen)
%% Let's call it...a...REVISION! 06/24/05 (semenko)
%% Yet again ... 06/24/06 (kaloyeh)
%% Revised 06/23/06 (sasen, quentin)
%% Revised 06/23/09 (quentin)
%% Tiny revisions 06/17/10 (milese); may do more
%% Revised 06/23/12 to add slide on BibTeX (belzner)

\documentclass[landscape]{slides}
\usepackage{graphics, color}
\usepackage{lscape}
\usepackage{standard-slide-include}
\usepackage{fancyhea}
\usepackage{lshort}
\title{Introduction to \LaTeX}
\author{RSI {\number\year} Staff}
\date{}

%\pagestyle{empty}

%\addtolength{\topmargin}{-1in}
%\addtolength{\textheight}{2in}
%\frenchspacing
%\hyphenpenalty=10000
%\rightskip=0pt plus1in

\begin{document}

% \newlength{\titlewd}
% \def\hzline{\makebox[\textwidth]{\hrulefill}}
% \def\mknewpage{\newpage}
% \def\titlesize{\large}
% \def\textsize{\normalsize}
% \def\mktitle#1{{
%  \def\thetitle{{\titlesize \bf #1}}
%  \settowidth{\titlewd}{\thetitle}
%  \flushleft \thetitle \hspace{-\titlewd}
%  \rule[-.3\baselineskip]{\textwidth}{.02in}
% }}
% \def\mkltitle#1{{
%  \def\thetitle{{\titlesize \bf #1}}
%  \settowidth{\titlewd}{\thetitle}
%  \flushleft \thetitle \hspace{-\titlewd}
%  \rule[.7\baselineskip]{\textwidth}{.02in}
% }}
% \def\i#1{\ifmmode\mbox{\it #1}\else{\it #1\/}\fi}
% \def\t#1{{\tt #1}}

\renewcommand{\labelitemii}{\labelitemi}
\renewcommand{\labelitemiv}{\labelitemi}

%\newcommand{\cmd}[1]{{\tt $\backslash$#1}}


%%% intro
\begin{slide}
%\vspace{3in}
%\begin{center}
%{\Large \bf Introduction to \LaTeX} \\
%\vspace{1in}
%{\large \bf RSI Staff} \\
%{\large \bf 27 June 19100}
%\end{center}
  \thispagestyle{empty}
  \maketitle
\end{slide}


%%% table of contents
\begin{slide}
  \begin{small}
    \tableofcontents
  \end{small}
\end{slide}

\addtocounter{slide}{-1}


\begin{slide}
\mktitle{What is \LaTeX?}
\begin{itemize}
\item Professional-quality typesetting program
\item Standard in scientific communication
\item Uses text-based commands to format
\end{itemize}

\begin{example}
\emph{This is very important.}
\end{example}

\begin{itemize}
\item Flexible: can format letters, papers, even books
\item Easy to effect changes to whole document
\end{itemize}
\end{slide}

% Note to future TAs: we're changing the order here because students
% want to get started; they don't want to hear us yammer on about what
% commands look like. So let's have them do an example and then explain
% at greater levels of detail afterwards. -- sasen, 2008

\begin{slide}
\mktitle{First Example}
\begin{itemize}
\item To begin, type
\begin{verbatim}
athena% cd ~/RSI/MiniPaper/
athena% gedit first.tex &
\end{verbatim}
\item Now type  the following in your document:
\begin{verbatim}
\documentclass{article}
\begin{document}
This is my first \LaTeX\ document.
\end{document}
\end{verbatim}
\end{itemize}
\end{slide}

\begin{slide}
  \mktitle{Compiling}
  \begin{itemize}
    \item To compile your document, save it and type at the prompt:
\begin{verbatim}
athena% make first.pdf
\end{verbatim}
    \item To look at it, type
\begin{verbatim}
athena% evince first.pdf &
\end{verbatim}
    \item You can also use \texttt{xpdf} or \texttt{acroread}
instead of \texttt{evince}
%(but \texttt{xpdf} won't work in 37-312).
\end{itemize}
\end{slide}

\begin{slide}
\mktitle{Starting your \LaTeX\ document}
\begin{itemize}
\item \cmd{documentclass}\{$\cdots$\} specifies the type of document being
written ({\tt article}, {\tt report}, {\tt beamer}, etc.) and takes various
options (font size, double spacing).

\item \cmd{usepackage}\{$\cdots$\} loads various features to your
document.
\item \cmd{begin\{document\}}$\cdots$\cmd{end\{document\}} surrounds
the body text of the document.
\end{itemize}
\end{slide}

\begin{slide}
\mktitle{\LaTeX\ command syntax} \LaTeX\ uses commands to indicate
document styles, formatting, etc.
\begin{itemize}
\item Commands begin with backslash {\tt $\backslash$}, required
arguments go in curly braces {\tt \{ \}}, optional ones in square
brackets {\tt [ ]}.
\item Many commands are given by \cmd{begin\{foo\}} ... \cmd{end\{foo\}},
which format the text between them based on {\tt foo}.
\end{itemize}
\begin{example}
\begin{center}
This text is centered.
\end{center}
\begin{equation}
\alpha \times \beta = \gamma
\end{equation}
\end{example}
\end{slide}

\begin{slide}
\mktitle{Adding a title/author}
\begin{itemize}
\item \cmd{title}\{$\cdots$\} and \cmd{author}\{$\cdots$\} are
used to specify the title and author of the paper.
\item \cmd{maketitle} then inserts the given information into the
  document and formats it appropriately.
\begin{verbatim}
\documentclass{article}
\title{Odyssey}
\author{Homer}
\begin{document}
\maketitle
This is my first \LaTeX\ document.
\end{document}
\end{verbatim}
\end{itemize}
\end{slide}

\begin{slide}
\mktitle{Typesetting: Spacing}
\begin{itemize}
\item Any number of whitespace characters is treated as one ``space''.
\item Any number of blank lines is treated as a paragraph break.
\begin{example}
It does not matter whether you
enter one or several     spaces
after a word.

An empty line starts a new
paragraph.
\end{example}
\end{itemize}
\end{slide}

\begin{slide}
\mktitle{Typesetting: Special characters}
\begin{itemize}
\item For quotes, use {\tt ``} (two backquotes) and {\tt ''} (two
apostrophes) instead of {\tt "}. For single quotes, just use
\texttt{`} and \texttt{'}.
\item However, emacs does the Right Thing\texttrademark\ (most of the time)
if you type {\tt "}, so you don't need to do anything
special.  \item {\tt \%} is the comment character.  Use \cmd{\%} to
insert a {\tt \%} in your document. If you just use {\tt \%}, \LaTeX\ 
will ignore the rest of that line.
\item Similarly, {\tt \_}, {\tt \$}, {\tt \#}, {\tt \&}, {\tt \{}, and
{\tt \}} all mean something special.  Use escapes like \cmd{\$} to
put these in your document.
\end{itemize}
\end{slide}



\begin{slide}
\mktitle{Layout: Sections} You can define sections in your paper.
\begin{verbatim}
\section{Introduction}
RSI is fun.

\subsection{First Part}
My mentor is nice.

\subsection{Second Part}
My counselor brought us ice cream!

\section{Conclusion}
This is the life.
\end{verbatim}
\end{slide}

\begin{slide}
\mktitle{Layout: Lists}
\begin{itemize}
\item Use {\tt itemize} to create bulleted lists and {\tt enumerate}
to create numbered lists:
\begin{example}
RSI kids do not
\begin{itemize}
\item set things on fire.
\item lose their room keys.
\item terrorize MIT students.
\end{itemize}
My plan for this summer is to:
\begin{enumerate}
\item Do some research.
\item Stay awake at mentorship.
\item Beat my roommate at Set.
\end{enumerate}
\end{example}
\end{itemize}
\end{slide}

\begin{slide}
\mktitle{Layout: Footnotes and Bibliography} \begin{itemize}

\item For more casual references, use footnotes:
\begin{example}
At some point during the summer,
the RSI students and the PROMYS~
\footnote{PROMYS sucks, we'll beat them this year.}
students will duke it out
on the ultimate field.
\end{example}
\item To refer to an outside source, cite it from your bibliography.
\begin{small}
\begin{verbatim}
Rumor has it that if you fall asleep in the lectures, you will be
assigned to cleaning all the bathrooms in Simmons.\cite{ugolini}

<biblio.bib>
@article {ugolini
...
}
\end{verbatim}
\end{small}
\end{itemize}
\end{slide}

\begin{slide}
\mktitle{Layout: References}
\begin{itemize}
\item \LaTeX\ lets you refer to one part of the paper from another
part.
\begin{verbatim}
\section{foo}
\label{sec:foo} Don't buy lots of Snapple on your meal card
to take to your rooms.  It makes the admins angry.
\end{verbatim}
$$\cdots$$

\begin{verbatim}
\section{bar}
In Section~\ref{sec:foo}, we discussed how you shouldn't
hoard lots of Snapple from LaVerde's.
\end{verbatim}
\item This will replace the \cmd{ref} command with foo's section number.
You can \cmd{label} almost anything, including sections, figures,
tables, and equations.
\end{itemize}
\end{slide}


\begin{slide}
\mktitle{RSI Papers: Setting Up}
You must do the following to work on your papers, 
\begin{enumerate}
\item Go to destination directory \\
    {\tt athena\% cd \~{ }/RSI/MiniPaper}

\item Put text and formatting instructions into a text file ({\tt
paper.tex}) using your choice of text editor.
\item To spellcheck, you can use \\ {\tt athena\% ispell -t {\it
filename}}
\end{enumerate}
\end{slide}

\begin{slide}
\mktitle{RSI Papers: Using the Templates} For your minipaper:
\begin{itemize}
\item  Edit {\tt abstract.tex}, {\tt biblio.bib}, {\tt cover.tex} (title),
and {\tt paper.tex} \\ Do NOT modify {\tt main.tex}! \vspace{-.25in}
\end{itemize}
\end{slide}

\begin{slide}
\mktitle{Latex Ouput: For Future Reference} There are various output
formats for your paper:
\begin{itemize}
\item Device independent (DVI) - standard output format of
\TeX\
\item Postscript (PS) - compiled version of document for
printing
\item Portable document format (PDF) - compressed version of
document for sharing
\end{itemize}

\begin{tabular}{|c|c|c|c|}
\hline & Creating & Viewing & Printing \\ & & &\\
\hline {\tt .dvi} & {\tt latex main.tex} & {\tt xdvi main.dvi} & {\tt dvips main.dvi} \\ & & &\\
\hline {\tt .ps} & {\tt dvips main.dvi -o} & {\tt gv main.ps} & {\tt lpr main.ps} \\ & & &\\
\hline {\tt .pdf} & \begin{minipage}{0.35\textwidth} \center{\tt pdflatex main.tex}\end{minipage} & {\tt acroread main.pdf} & Print from {\tt xpdf} \\ & & &\\
\hline\end{tabular}
\end{slide}

% !!!
% We need to see if xpdf actually works in the one cluster; if it
% doesn't, we need to change the previous slide

\begin{slide}
\mktitle{What does that ``make'' command do?}
\begin{itemize}
  \item ``make'' reads a file called ``Makefile'' from the current
    directory and uses it to describe how to create a file.
  \item We've provided a makefile containing:
\begin{verbatim}
include /mit/rsi/misc/texenv.make

all: main.pdf

%.pdf: %.tex
rubber --pdf $<

.PHONY: clean
clean:
rm *.aux *.log *.dvi
\end{verbatim}
  \item This says to use \verb+rubber --pdf $<+ to select the necessary commands to compile {\tt .tex} and related files into {\tt .pdf} files.
\end{itemize}
\end{slide}

\begin{slide}
\mktitle{Errors}
Sometimes things will go wrong.  When you get an error:
\begin{itemize}
\item Look for the line number.  {\tt M-x goto-line} in emacs will take
you there.
\item Hit {\tt x} to stop the compiler.
\item Hit {\tt return} to tell the compiler to continue past the error.
\end{itemize}
Your papers must compile cleanly when you submit them, without having to
press return.
\end{slide}

\begin{slide}
% Actually four, counting the minus sign in math mode
\mktitle{Typesetting: The Hyphen and Its Friends} The hyphen
character (\texttt{-}) is used to generate three distinct
punctuation marks.
\begin{itemize}
\item Hyphen (in a compound word):
just \texttt{-}, as in \texttt{path-homotopic}.
\item En-dash (in a numerical range):
\texttt{--}, as in \texttt{1984--2008}.
\item Em-dash (break in a sentence):
\texttt{---}, as in \texttt{Yes---just not today} or \texttt {No---I
  just tested positive}.
\begin{example}
That absent-minded TA threw
pages 8--12 away---what a fool!
\end{example}
\end{itemize}
\end{slide}

\begin{slide}
\mktitle{Typesetting: Hacking Spacing} There are various ways to force
\LaTeX\ to create spaces:
\begin{itemize}

\item \cmd{ } (backslash-space) and \verb=~= force a space after a command: the latter prevents a linebreak
(e.g., \verb=In Section~2, we ...=)

\item \cmd{$\backslash$} forces a linebreak.  Don't use this unless
you're really sure!

\item \cmd{hspace\{}\#\#\#\texttt{\}} and
\cmd{vspace\{}\#\#\#\texttt{\}} force horizontal and vertical spaces
respectively, with size \#\#\# (a positive, real number with units of length,
e.g., \verb=\hspace{0.5cm}=).

\end{itemize}
\end{slide}


\begin{slide}
\mktitle{Typesetting: Hacking Text Formatting} \LaTeX\ is all about not
worrying about formatting. But {\it very} occasionally it's useful
to have different font attributes.

\begin{itemize}
\item Various font adjustments:
\begin{example}
This is \emph{emphasis}.
This is \textbf{bold}.
This is \texttt{teletype}.
\end{example}
\item Various font sizes:
\begin{example}
This is {\large large}.
This is {\Large Large}.
This is {\LARGE LARGE}.
\end{example}
\end{itemize}
\end{slide}

\begin{slide}
\mktitle{Using BibTeX}
The template is in the file biblio.bib and looks like below. Fill in the spaces between the curly braces with the proper information.

\begin{verbatim}
@article{name,
	 author={   },
	 title={   },
	 journal={   },
	 volume={   },
	 year = {   },
	 pages = {   }
	}
\end{verbatim}

You will have access to three BibTeX templates (i.e., Article, Book,
Website) in biblio.bib. Copy them as needed.
\end{slide}

\begin{slide}
\mktitle{Getting Help}
\begin{itemize}
\item For frequently asked questions and answers, \hfill\\{\tt http://web.mit.edu/rsi/www/} and follow the help link.
\item {\tt rsi-help} zephyr class.  Use the {\tt zrsihelp}
command.
\item A student association known as SIPB also offers advice at
\[\hbox{\tt http://www.mit.edu/sipb/docs.html}.\]
Its office is located just outside the w20-575 cluster.
Also,  SIPB's ``Inessential \LaTeX,'' found at
\[\hbox{\texttt{/mit/sipb/doc/iLaTeX.PS}},\] is a useful reference.
\item ``The Not So Short Introduction to \LaTeX'' is at
\[\hbox{\tt http://tobi.oetiker.ch/lshort/lshort.pdf}\]
\end{itemize}
\end{slide}

\end{document}

\documentclass[13pt,xcolor=table]{beamer}
%\documentclass[handout,xcolor=table]{beamer}  % For printable version

\usepackage{graphicx}             % for Windows
\usepackage{array}
\usepackage{xmpmulti}

% \usepackage{colortbl}
% \usepackage[table]{xcolor}
% \usepackage{beamerthemesplit}

% \usenavigationsymbolstemplate{} % eliminates navigation symbols

%
% PRESENTATION THEMES
%
% \usetheme{default}        % No Navigation Bars
% \usetheme{Bergen}
% \usetheme{Boadilla}
% \usetheme{AnnArbor}       % similar to Boadilla
% \usetheme{CambridgeUS}    % similar to Boadilla
% \usetheme{Madrid}
% \usetheme{Pittsburgh}
% \usetheme{Rochester}
% \usetheme{Antibes}        % Tree-Like Navigation Bar
% \usetheme{JuanLesPins}    % similar to Antibes
% \usetheme{Montpellier}
% \usetheme{Berkeley}       % Table of Contents Sidebar
% \usetheme{PaloAlto}       % similar to Berkeley
% \usetheme{Goettingen}
% \usetheme{Marburg}        % similar to Goettingen
% \usetheme{Hannover}
% \usetheme{Berlin}         % Mini Frame Navigation
% \usetheme{Ilmenau}        % similar to Berlin
% \usetheme{Dresden}        % similar to Berlin
% \usetheme{Darmstadt}
% \usetheme{Frankfurt}      % similar to Darmstadt
% \usetheme{Singapore}
% \usetheme{Szeged}
\usetheme{Copenhagen}     % Section and Subsection Tables
% \usetheme{Luebeck}        % similar to Copenhagen
% \usetheme{Malmoe}         % similar to Copenhagen
% \usetheme{Warsaw}         % similar to Copenhagen

%
% INNER THEMES
%
% \useinnertheme{circles} 
% \useinnertheme{rectangles}
% \useinnertheme[shadow]{rounded} 
% \useinnertheme{inmargin} 

%
% OUTER THEMES
%
% \useoutertheme{infolines} 
% \useoutertheme[]{miniframes} 
% \useoutertheme[]{smoothbars} 
% \useoutertheme[]{sidebar} 
% \useoutertheme{split}
% \useoutertheme{shadow} 
% \useoutertheme[]{tree} 
% \useoutertheme{smoothtree} 


%
% COLOR THEMES
%
% \usecolortheme{seahorse}
% \usecolortheme{albatross}
% \usecolortheme{beaver}
% \usecolortheme{beetle}
% \usecolortheme{crane}
% \usecolortheme{dolphin}
% \usecolortheme{dove}
% \usecolortheme{fly}
% \usecolortheme{lily}
% \usecolortheme{orchid}
% \usecolortheme{rose}
% \usecolortheme{seagull}
% \usecolortheme{sidebartab}
% \usecolortheme{structure}
% \usecolortheme{whale}
% \usecolortheme{wolverine}

%
% FONT THEMES
% 
% \usefonttheme{serif} 
% \usefonttheme{structureitalicserif} 
% \usefonttheme{structurebold} 
% \usefonttheme{structuresmallcapsserif} 

% \mode<presentation>

\let\Tiny=\tiny             % Gets rid of Font shape `OT1/cmss/m/n' warning

\title[Intro to Beamer]{An Introduction to \LaTeX\ Beamer}
\subtitle[Slides]{Making Slide Presentations}
\author[S Chang]{Silva Chang}
\institute[CU-Boulder]{CU-Boulder Department of Applied Mathematics}
\date[Sum09]{Summer 2009}

\begin{document}

\frame{
  \titlepage
}

\begin{frame}[shrink]
  \frametitle{Table of Contents} 
  
    \tableofcontents[pausesections] 
\end{frame}

\section{Introduction}
\subsection{Frames}

\begin{frame}[fragile]
  \frametitle{Basic Commands}
  
  A Beamer file looks like:
  
  \begin{block}{}    
    \begin{verbatim}
      \documentclass{beamer}
      
      \begin{document}
        ...
      \end{document}
    \end{verbatim}
  \end{block}
\end{frame}


\begin{frame}[fragile]
  \frametitle{Basic Commands}
  
  A frame can be created with either of the following:
  
  \begin{block}{}   
    \begin{verbatim}

    \begin{frame}
      ...
    \end{frame}
  \end{verbatim}
  \end{block}
  
  \begin{block}{}  
    \begin{verbatim}

    \frame {
      ...
    }
  \end{verbatim}
  \end{block}
\end{frame}

\begin{frame}[fragile]
  \frametitle{Basic Commands}
  
  You can use the default layout or choose a different theme:
  
  \begin{block}{}    
    \begin{verbatim}
      \usetheme{Berkeley}
      \usecolortheme{wolverine}
    \end{verbatim}
  \end{block}
\end{frame}


\subsection{Title Page, Table of Contents}

\begin{frame}[fragile]
  \frametitle{Basic Commands}
  
  Add info about the title, author, etc., to the preamble:
  
  \begin{block}{} 
    {\footnotesize   
    \begin{verbatim}
      \title    [Beamer Intro]{An Introduction to \LaTeX\ Beamer}
      \subtitle [Slides]      {Making Slide Presentations}
      \author   [Workshop]    {\LaTeX\ Workshop}
      \institute[CU-Boulder]  {University of Colorado at Boulder}
      \date     [Sum09]       {Summer 2009}
    \end{verbatim}
    }
  \end{block}
\end{frame}

\begin{frame}[fragile]
  \frametitle{Basic Commands}
  
  Then create a title page in the document:
  
  \begin{block}{}    
    \begin{verbatim}
      \frame {
        \titlepage
      }
    \end{verbatim}
  \end{block}
\end{frame}

\begin{frame}[fragile]
  \frametitle{Basic Commands}
  
  If you use sectioning commands in your document:
  
  \begin{block}{}    
    \begin{verbatim}
      \section{First Section}
      \subsection{First Subsection}
      
      \frame {
        ...
      }

    \end{verbatim}
  \end{block}
\end{frame}

\begin{frame}[fragile]
  \frametitle{Basic Commands}
  
  You can create a table of contents:
  
  \begin{block}{}    
    \begin{verbatim}
      \frame {
        \tableofcontents
      }
    \end{verbatim}
  \end{block}
  
  \begin{block}{}    
    \begin{verbatim}
      \frame {
        \tableofcontents[pausesections] 
      }
    \end{verbatim}
  \end{block}
\end{frame}


\subsection{Blocks}

\begin{frame}[fragile]
  \frametitle{Basic Commands}
  
  To create a block in your frame:
  
  \begin{block}{}    
    \begin{verbatim}
      \begin{block}{}
        ...
      \end{block}
    \end{verbatim}
  \end{block}
  
  \begin{block}{Block Title}    
    \begin{verbatim}
      \begin{block}{Block Title}
        ...
      \end{block}
    \end{verbatim}
  \end{block}
\end{frame}

\begin{frame}[fragile]
  \frametitle{Basic Commands}
  
  Most \LaTeX\ commands work in Beamer. For example,
  
  \begin{block}{}    
    \begin{verbatim}
      \[ a^2 + b^2 = c^2. \]
    \end{verbatim}
  \end{block}
  produces \[ a^2 + b^2 = c^2. \]

\end{frame}

\subsection{Options}

\begin{frame}[fragile,plain]
  \frametitle{Basic Commands}
  The {\tt frame} environment accepts optional arguments:
  \begin{itemize}
    \item {\tt plain}. No headers or footers.
    \item {\tt shrink}. Shrink contents to fit on one slide.
    \item {\tt squeeze}. Reduce vertical space.
    \item {\tt allowframebreaks}. Split slide if necessary.
  \end{itemize}
  
  \vspace{1em}
  
  \begin{block}{}    
    \begin{verbatim}
      \begin{frame}[plain]
        ...
      \end{frame}
    \end{verbatim}
  \end{block}
\end{frame}

\begin{frame}[fragile]
  \frametitle{Basic Commands}
  
  To eliminate the navigation symbols, add to the preamble:
  
  \begin{block}{}    
    \begin{verbatim}
      \usenavigationsymbolstemplate{}
    \end{verbatim}
  \end{block}  
\end{frame}

\begin{frame}[fragile]
  \frametitle{Basic Commands}
  
  To create a printout of your slides, add \verb+handout+ to the \verb+\documentclass+ options:
  
  \begin{block}{}    
    \begin{verbatim}
      \documentclass[handout]{beamer}
    \end{verbatim}
  \end{block}
  \begin{block}{}    
    \begin{verbatim}
      \only<beamer>{To appear in slides only.}
      \only<handout>{To appear in handout only.}
    \end{verbatim}
  \end{block}
\end{frame}

\section{Using Overlays}

\subsection{Text}

\begin{frame}[fragile]
  \frametitle{Uncovering Text Piecewise}
  
  To uncover text gradually, use the \verb+\pause+ or \verb+\onslide+ command.
  
  \begin{block}{}    
    \begin{verbatim}
      Line 1 \pause
      Line 2 \pause
      Line 3
    \end{verbatim}
  \end{block}
  
  \begin{block}{}    
    \begin{verbatim}
      \onslide<1>{Text 1} 
      \onslide<2>{Text 2}
      \onslide<3>{Text 3}      
    \end{verbatim}
  \end{block}
\end{frame}

\frame
{
  \frametitle{Uncovering Text Piecewise}
  \setbeamercovered{transparent}
  
  \begin{center}
  The default size of a Beamer slide\\ \pause
  is 128 mm by 96 mm\\ \pause
  or 5.04 in by 3.78 in.
  \end{center}
}

\frame
{
  \frametitle{Uncovering Text Piecewise}
  \setbeamercovered{transparent}
  
  \begin{center}
    \onslide<1>{see no evil,} 
    \onslide<2>{hear no evil,}
    \onslide<3>{speak no evil.}  
  \end{center}
}

\begin{frame}[fragile]
  \frametitle{Uncovering Text Piecewise}
  
  The \verb+\onslide+ command allows you to specify which text goes on each overlay.
  
  \begin{block}{}    
    \begin{verbatim}
        \onslide<4>   Overlay  4 only
        \onslide<1,3> Overlays 1 and 3 only
        \onslide<2-4> Overlays 2 through 4
        \onslide<2->  Overlay  2 until the end
        \onslide<-3>  Overlays 1 through 3
    \end{verbatim}
  \end{block}
\end{frame}

\frame
{
  \frametitle{Uncovering Text Piecewise}
  \setbeamercovered{transparent}
  
  \begin{center}
    \onslide<2>{Multiples of 2:}\\
    \onslide<3>{Multiples of 3:}\\
    \onslide<4>{Multiples of 6:}\\[1em]
    
    \onslide<1>{1} \quad \onslide<1-2>{2}  \quad \onslide<1,3>{3}  \quad\onslide<1-2>{4} \quad
    \onslide<1>{5} \quad \onslide<1-4>{6}\quad \onslide<1>{7}  \quad\onslide<1-2>{8} \quad
    \onslide<1,3>{9} \quad \onslide<1-2>{10} \quad \onslide<1>{11} \quad\onslide<1-4>{12}
  \end{center}
}

\subsection{Items}

\begin{frame}[fragile]
  \frametitle{Uncovering Text Piecewise}
  
  To reveal items gradually, use the \verb?\begin{itemize}[<+->]? command.
  
  \begin{block}{}    
    \begin{verbatim}
      \begin{itemize}[<+->] 
        \item   First item
        \item   Second item
        \item   Third item
      \end{itemize}
    \end{verbatim}
  \end{block}
\end{frame}

\frame
{
  \frametitle{Uncovering Items Piecewise}
  \setbeamercovered{transparent}

  \begin{itemize}[<+->] 
    \item   $\alpha$ alpha
    \item   $\beta$ beta
    \item   $\gamma$ gamma
  \end{itemize}
}

\begin{frame}[fragile]
  \frametitle{Uncovering Several Items at a Time}
  
 As with \verb+\onslide+, you can indicate which overlays you would like an item to appear in.
  
  \begin{block}{}    
    \begin{verbatim}
      \begin{itemize}
        \item<3>   Slide  3 only
        \item<1,4> Slides 1 and 4 only
        \item<2->  Slide  2 until the end
      \end{itemize}
    \end{verbatim}
  \end{block}
\end{frame}

\frame
{
  \frametitle{Uncovering Several Items at a Time}
  \setbeamercovered{transparent}
  
  These numbers are \onslide<1>{even,} \onslide<2>{odd,} \onslide<3>{prime.}
  \begin{itemize}
  \item<1,3> 2
  \item<2-> 3 
  \item<1> 4
  \item<2-3> 5
  \item<1> 6 
  \end{itemize} 
}

\begin{frame}[fragile]
  \frametitle{Uncovering Several Items at a Time}
  
 By default hidden overlays are not visible. To toggle between transparent and invisible, use these commands: 
  
  \begin{block}{}    
    \begin{verbatim}
      \setbeamercovered{transparent}
    \end{verbatim}
  \end{block}
  
  \begin{block}{}    
    \begin{verbatim}
      \setbeamercovered{invisible}
    \end{verbatim}
  \end{block}
\end{frame}

  
\frame
{
  \frametitle{Uncovering Text Piecewise}
  
  \begin{center}
    \onslide<2>{Multiples of 2:}\\
    \onslide<3>{Multiples of 3:}\\
    \onslide<4>{Multiples of 6:}\\[1em]
    
    \onslide<1>{1} \quad \onslide<1-2>{2}  \quad \onslide<1,3>{3}  \quad\onslide<1-2>{4} \quad
    \onslide<1>{5} \quad \onslide<1-4>{6}\quad \onslide<1>{7}  \quad\onslide<1-2>{8} \quad
    \onslide<1,3>{9} \quad \onslide<1-2>{10} \quad \onslide<1>{11} \quad\onslide<1-4>{12}
  \end{center}
}

\begin{frame}[fragile]
  \frametitle{Replacing Text}
  
  To overlay text \emph{on top of} existing text, use \verb+\only+:
  
  \begin{block}{}    
    \begin{verbatim}
      \only<1> {Overlay 1}
      \only<2> {Overlay 2}
      \only<3> {Overlay 3}
    \end{verbatim}
  \end{block}
\end{frame}

\frame
{
  \frametitle{Replacing Text}
  These numbers are \only<1>{even}\only<2>{odd}\only<3>{prime}.
  \begin{itemize}
  \item<1,3> 2
  \item<2-> 3 
  \item<1> 4
  \item<2-3> 5
  \item<1> 6 
  \end{itemize} 
}

\frame
{
  \frametitle{Replacing Text}
  This is the \only<1>{\tt 1st}\only<2>{\tt 2nd}\only<3>{\tt 3rd} overlay. 
}

\subsection{Highlighting}

\begin{frame}[fragile]
  \frametitle{Highlighting an Item}  
  Use the \verb+\alert+ command to highlight an item:  
  \begin{block}{}    
    \begin{verbatim}
      This is \alert{highlighted text.}
    \end{verbatim}
  \end{block}
\end{frame}

\begin{frame}[fragile]
  \frametitle{Highlighting an Item}
      This is \alert{highlighted text}.  
\end{frame}

\begin{frame}[fragile]
  \frametitle{Highlighting Items Piecewise}
  
  Highlight items piecewise by using the following command:  
  \begin{block}{}    
    \begin{verbatim}
      \begin{itemize}[<+-| alert@+>] 
        \item First point. 
        \item Second point. 
        \item Third point. 
      \end{itemize} 
    \end{verbatim}
  \end{block}
\end{frame}

\frame
{
  \frametitle{Highlighting Items Piecewise} 
  
  \begin{itemize}[<+-| alert@+>] 
    \item $\sin \theta$
    \item $\cos \theta$
    \item $\tan \theta$
  \end{itemize} 
}

\begin{frame}[fragile]
  \frametitle{Highlighting Two Items at a Time}
  
  Or indicate which items are highlighted on which overlay.
  \begin{block}{}    
    \begin{verbatim}
      \begin{itemize} 
        \item<1-| alert@1> First overlay. 
        \item<2-| alert@2> Second overlay. 
        \item<1-| alert@1> First overlay. 
      \end{itemize} 
    \end{verbatim}
  \end{block}
\end{frame}

\frame
{
  \frametitle{Highlighting Two Items at a Time} 
    \begin{itemize} 
      \item<1-| alert@1> 2, 4, 6, 8
      \item<2-| alert@2> 3, 6, 9, 12
      \item<1-| alert@1> 4, 8, 16, 20
    \end{itemize} 
}
  
\begin{frame}[fragile]
  \frametitle{Applying Effect Using Overlays}
  
  To apply an effect using overlays, add \texttt{<}\emph{slide num}\texttt{>} to the command:
  \begin{block}{}    
    \begin{verbatim}
      \cmd<1>{Overlay 1}
      \cmd<2>{Overlay 2}
      \cmd<3>{Overlay 3}
    \end{verbatim}
  \end{block}
\end{frame}

\frame
{ 
  \frametitle{Applying Effects Using Overlays}
  \textbf{Bold} on all three overlays. \\
  \textbf<2>{Bold} on the second overlay only. \\
  \textbf<3>{Bold} on the third overlay only. 
}

 
\subsection{Equations}
  
\begin{frame}[fragile]
  \frametitle{Uncovering Equations Piecewise}

  \begin{block}{}    
    \begin{verbatim}
      \begin{gather} 
        Equation 1 \\ 
        \onslide<2->{Equation 2 \\} 
        \onslide<3->{Equation 3 \\} 
        \notag 
      \end{gather} 
    \end{verbatim}
  \end{block}
\end{frame}

\frame
{
  \frametitle{Uncovering {\tt gather} Equations Piecewise} 
  \begin{gather} 
    \sin \theta = 1/2 \\ 
    \onslide<2->{\cos \theta = {\sqrt 3} / 2 \\[.2em]} 
    \onslide<3->{\tan \theta = \dfrac{\sin \theta}{\cos \theta} = {\sqrt 3} / 3 \\} 
    \notag 
  \end{gather} 
}

\begin{frame}[fragile]
  \frametitle{Uncovering {\tt align} Equations Piecewise}

  \begin{block}{}    
    \begin{verbatim}
      \begin{align} 
        ... &= ... \\ 
        \uncover<2->{&= ... \\} 
        \uncover<3->{&= ... \\} 
        \notag 
      \end{align} 
    \end{verbatim}
  \end{block}
\end{frame}

\frame
{
  \frametitle{Uncovering {\tt align} Equations Piecewise} 
%  \setbeamercovered{invisible}        % necessary for Windows
  \begin{align} 
    \cos 2\theta &= \cos^2 \theta - \sin^2 \theta \\ 
    \uncover<2->{&= 2\cos^2 \theta - 1 \\} 
    \uncover<3->{&= 1 - 2\sin^2 \theta \\} 
    \notag 
  \end{align} 
}

  
\subsection{Tables}
  
\begin{frame}[fragile]
  \frametitle{Uncovering a Table Rowwise} 
  Use \verb+\pause+ to uncover a table one row at a time

  \begin{block}{}    
    \begin{verbatim}
      \begin{tabular}{...}
        row 1 \pause \\
        row 2 \pause \\
        row 3
      \end{tabular}  
    \end{verbatim}
  \end{block}
\end{frame}

\frame
{
  \frametitle{Uncovering a Table Rowwise} 

  {\Large
  \begin{center}
    \begin{tabular}{ccccc}
    1 \pause \\
    1 & 2 & 1 \pause \\
    1 & 3 & 3 & 1 \pause \\
    1 & 4 & 6 & 4 & 1
    \end{tabular} 
  \end{center}
  }
}

\frame
{
  \frametitle{Uncovering an Array Rowwise} 
  \begin{center}
    \rowcolors[]{1}{violet!40}{violet!30}
    $\begin{array}{c!{\vrule}c!{\vrule}c!{\vrule}c}   % need array package
      \theta & \sin \theta & \cos \theta & \tan \theta \pause \\\hline
      0 & 0 & 1 & 0 \pause \\
      \pi/2 & 1 & 0 & und \pause \\
      \pi & 0 & -1 & 0\pause \\
      3\pi/2 & -1 & 0 & und
    \end{array} $   
  \end{center}
}

\begin{frame}[fragile]
  \frametitle{Uncovering a Table Columnwise} 
  Use \verb+\onslide+ to uncover a table one column at a time

  \begin{block}{}    
    \begin{verbatim}
      \begin{tabular}{c<{\onslide<2->}c
                       <{\onslide<3->}c
                       <{\onslide}c}
        row 1 \\
        row 2 \\
        row 3
      \end{tabular}  
    \end{verbatim}
  \end{block}
\end{frame}
  
\frame
{
  \frametitle{Uncovering a Table Columnwise} 

  {\Large
  \begin{center}
    \begin{tabular}{c<{\onslide<2->}c
                     <{\onslide<3->}c
                     <{\onslide<4->}c
                     <{\onslide}c}
    1 & 0 & 0 & 0 \\
    0 & 1 & 0 & 0 \\
    0 & 0 & 1 & 0 \\
    0 & 0 & 0 & 1
  \end{tabular}
  \end{center}
  }
}
  
\frame
{
  \frametitle{Uncovering an Array Columnwise} 
  \begin{center}
    \rowcolors[]{1}{violet!40}{violet!30}
  
    $\begin{array}{c<{\onslide<2->}!{\vrule}c
                       <{\onslide<3->}!{\vrule}c
                       <{\onslide<4->}!{\vrule}c
                       <{\onslide}c}
      \theta & \sin \theta & \cos \theta & \tan \theta \\
      0 & 0 & 1 & 0 \\
      \pi/2 & 1 & 0 & und \\
      \pi & 0 & -1 & 0 \\
      3\pi/2 & -1 & 0 & und
    \end{array} $   
  \end{center}
}  
  
\begin{frame}[fragile]
  \frametitle{Uncovering a Table Elementwise} 
  Use \verb+\onslide+ to uncover a table one element at a time

  \begin{block}{}    
    \begin{verbatim}
      \begin{tabular}{c c} 
        \onslide<1->{A} & \onslide<2->{B}  \\
        \onslide<3->{C} & \onslide<4->{D}  \\ 
      \end{tabular}  
    \end{verbatim}
  \end{block}
\end{frame}
  
% \frame {
%   \frametitle{Uncovering a Table Elementwise} 
%   \setbeamercovered{invisible}
%   {\Huge
%     \begin{center} 
%       \begin{tabular}{c c} 
%         \onslide<1->{A} & \onslide<2->{B}  \\
%         \onslide<3->{C} & \onslide<4->{D}  \\ 
%       \end{tabular}
%     \end{center} 
%   }
% }
  
  
\frame
{ 
  \frametitle{Uncovering a Table Elementwise} 
  {\Huge 
  \begin{center} 
  \begin{tabular}{c|c|c} 
  \onslide<6->{O} & \onslide<3->{X} & \onslide<5->{X} \\ \hline 
  \onslide<8->{O} & \onslide<2->{O} &  \\ \hline 
  \onslide<4->{O} &  & \onslide<7->{X} 
  \end{tabular} 
  \end{center} 
  } 
}
 
  
\section{Graphics}
  
\begin{frame}[fragile]
  \frametitle{Inserting Graphics} 
  Use \verb+\includegraphics+ to insert an image.

  \begin{block}{}    
    \begin{verbatim}
      \includegraphics<1>{...}
      \includegraphics<2>{...}
    \end{verbatim}
  \end{block}
\end{frame}
  
  
\frame
{
  \frametitle{Inserting Graphics} 
  \begin{center}
  \includegraphics<1>[scale=.25]{snowflake.jpg}% 
  \includegraphics<2>[scale=.75]{snowflake.jpg}% 
  \end{center}
}

  
\frame
{
 
  \frametitle{Sierpinski Triangles} 
  \begin{center} 
  \multiinclude[format=jpg,width=3in]{sierp}  % need package xmpmulti
  \end{center} 
}
 
  
  
\section{Block Environments}
  
\begin{frame}[fragile]
  \frametitle{Block Environments} 
  Beamer provides special block environments

  \begin{block}{}    
    \begin{verbatim}
      \begin{env}
        ...
      \end{env}
    \end{verbatim}
  \end{block}
  
  where {\tt env} is one of the following:\\[.5em]
  {\tt definition},\ \ {\tt theorem},\ \ {\tt lemma},\ \ {\tt corollary},\ \ {\tt proof},\ \ or {\tt alertblock}.
\end{frame}

\subsection{Theorems and Definitions}
  
\frame
{ 
  \frametitle{Definitions and Examples}
  \begin{definition} 
  The \alert{indefinite integral} $\int f(x)\,dx$ is the set of all antiderivatives \\of $f$. 
  \end{definition} 
  \pause
  
  \begin{example} 
  \begin{itemize} 
  \item $\int 2x\,dx = x^2 + C$
  \pause  
  \item $\int 3x^2\,dx = x^3 + C$ 
  \end{itemize} 
  \end{example} 
}

  
\frame
{ 
  \frametitle{Theorems} 
  \framesubtitle{and Proofs} 
  \begin{theorem} 
     \[ \lim_{\theta \to 0} \dfrac{\sin \theta}{\theta} = 1 \] 
  \end{theorem} 
  \pause
  \begin{proof} 
    \begin{align*}
      \lim_{\theta \to 0} \dfrac{\sin \theta}{\theta} = \lim_{\theta \to 0} \dfrac{\cos \theta}{1} = \cos 0 = 1.
    \end{align*}
  \end{proof} 
  \pause
  By L'H\^{o}pital's Rule.
}
 
  
\frame
{
    \frametitle{Lemma and Corollary}
    \framesubtitle{and Alertblock}
     \begin{lemma}
      A lemma.
    \end{lemma}
    \pause
    
    \begin{corollary}
      A corollary.
    \end{corollary}
    \pause
    
   \begin{alertblock}{Title}
      An alertblock.
    \end{alertblock}
}

  
\subsection{Columns}

\begin{frame}[fragile]
  \frametitle{Columns}
  
  You can create columns of text. 

  \begin{block}{}    
    \begin{verbatim}
      \begin{columns}
        \column{...} 
          ...
        
        \column{...} 
          ... 
      \end{columns} 
    \end{verbatim}
  \end{block}

\end{frame}

\frame
{
  \frametitle{Columns} 
  \framesubtitle{And Citations}
  
  \begin{columns}[t]
    \column{.5\textwidth} 
      \begin{block}{\TeX} 
        Created by Donald Knuth.
        \cite{texbook}
      \end{block} 
    
    \pause
    
    \column{.5\textwidth} 
      \begin{block}{\LaTeX} 
        Created by Leslie Lamport.
        \cite{latexbook} 
      \end{block} 
  \end{columns} 
}
  
\begin{frame}[fragile]
  \frametitle{Columns}
  
  And create citations using bibliography ``keys''.

  \begin{block}{}    
    \begin{verbatim}
        \cite{key1} 
        
        \begin{thebibliography}
          \bibitem{key1}
          ...
        \end{thebibliography}
    \end{verbatim}
  \end{block}

\end{frame}

\subsection{Code}

\begin{frame}[fragile]
  \frametitle{Inserting Code}
  
  Use the {\tt semiverbatim} environment to insert code fragments and highlight parts of the code. 

  \begin{block}{}    
    \begin{verbatim}
      \begin{frame}[fragile]
        \begin{semiverbatim} 
          \onslide<1->{ ... }
          \onslide<2->{ \alert<2>{...} }
          \onslide<3->{ ... }
        \end{semiverbatim}
      \end{frame}
    \end{verbatim}
  \end{block}  
\end{frame}


\begin{frame}[fragile] 
  \frametitle{Inserting Code} 
  \setbeamercovered{transparent}
  
  \begin{semiverbatim} 
     \onslide<1->{\\begin\{align\} }
     \onslide<2->{  \alert<2>{f(x) \alert<5>{&=} x(x-1)(x+1) \\\\ } }
     \onslide<3->{  \alert<3>{     \alert<5>{&=} x(x^2 - 1) \\\\} }
     \onslide<4->{  \alert<4>{     \alert<5>{&=} x^3 - x \\\\} }
     \onslide<4->{  \alert<4>{\\notag } }
     \onslide<1->{ \\end\{align\} }          
  \end{semiverbatim}
\end{frame}
 
  
\section{Bibliography} 
  
\begin{frame}[fragile]
  \frametitle{Bibliography}
  
  Use the {\tt thebibliography} environment to create a bibliography.

  \begin{block}{}    
    \begin{verbatim}
      \begin{thebibliography}
        \setbeamertemplate{bibliography item}[book]
        
        \bibitem[abbr1]{key1}
        Author1.
        \newblock Title1.
        \newblock Publisher1, year1.
        
        \bibitem[abbr2]{key2}
        ...
      \end{thebibliography}
    \end{verbatim}
  \end{block}

\end{frame}

\frame
{
  \frametitle{Bibliography}

  \begin{thebibliography}{} 
   
  \setbeamertemplate{bibliography item}[book]
  \bibitem[Knu84]{texbook} 
  Donald E. Knuth. 
  \newblock {\em The {\TeX}book}. 
  \newblock Addison-Wesley, 1984.
  
  \bibitem[Lam94]{latexbook} 
  Leslie Lamport. 
  \newblock {\em {\LaTeX}: A Document Preparation System}.
  \newblock Addison-Wesley, 1994.
  
  \setbeamertemplate{bibliography item}[article]  
  \bibitem[HL24]{hardylittlewood} 
  G.H. Hardy and J.E. Littlewood. 
  \newblock A Further Contribution to the Study of Goldbach's Problem. 
  \newblock {\em Proceedings of the  London Mathematical Society}, 2(22):44--56, 1924.
  \end{thebibliography}
}


\end{document}
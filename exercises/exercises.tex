\begin{center}
\centerline{\LaTeX{} Workshop Folha de Exercicios}
\end{center}

%% -------------------------------------------------------------------------------------------------
\section{\LaTeX{}} 
\hspace{1pt}
\vspace{1pt}

Please check references~\cite{oetiker2010not,forbes2011documentation} for more information.


\section{Exercicios}
\hspace{1pt}

\vspace{1pt}

\begin{enumerate}
\item Comece por repetir os exemplos apresentados nos slides deste workshop;
\item Descompacte o ficheiro ``latex\_workshop.zip''. Irá encontrar uma pasta com o nome ``Ex1'' com um conjunto de ficheiros incluídos. Abra o ficheiro article.tex com o TeXworks e resolva as seguintes questões:
\begin{enumerate}
  \item\label{one:one} No \verb!documentclass!: definir estilo para que o artigo siga o formato da \emph{IEEE transactions}\\ (\verb!\documentclass{IEEEtran}!);
  \item Defina o título (\verb!\title{...} ... \maketitle!);
  \item Defina o autor (por exemplo, o seu nome);
  \item Acrescente mais dois autores (por exemplo, o seu nome e de um dos seus colegas);
  \item\label{one:five} Crie o ambiente para o resumo (\verb!\begin{abstract}...\end{abstract}!);
  \item Incorpore o módulo Lorem Ipsum (\verb!\usepackage{lipsum}!);
  \item Popule o $abstract$ graças ao módulo Lorem Ipsum com dois parágrafos (consulte a documentação do ctan);
  \item Crie 5 Secções (Introdução, Estado da Arte, Métodos, Discussão e Conclusão);
  \item Adicone o módulo \verb!utf8!;
  \item Popule as secções criadas com o Lorem Ipsum à descrição;
  \item Incorpore o módulo para a utilização de imagens (\verb!\usepackage{graphicx}!);
  \item Incorpore o módulo para as sub-figuras (\verb!\usepackage{subfig}!);
  \item Crie uma grelha de 2 linhas por 5 columas para as imagens que se encontra na pasta \emph{imgs};
  \item Experimente utilizar \verb!\begin{figure*} ... \end{figure*}! em vez de \verb!\begin{figure} ... \end{figure}! Discuta o resultado.
  \item Legende cada uma das fíguras individuais;
  \item Crie um $label$ numa das subfiguras e referencie no texto;
  \item Legende a figura global;
  \item Crie um $label$ e referencie a figura no texto (\verb!\label{}...\ref{}!);
  \item Crie uma lista de items de Flores com uma $label$ associado a cada item;
  \item Referencie os items no texto; 
 \item Cite o artigo que se encontra no ficheiro $refs.bib$ (Porque não funcionou?); 
  \item Adicione a secção para as referências;
  \item No estilo da bibliografia, utilize o estilo \verb!IEEEtran! (Reparou na diferença?).
  \item Crie uma entrada bibliográfica (tipo artigo) com os seguintes campos:
    \begin{description}
      \item[author] Bishop, Tom E and Favaro, Paolo
      \item[title] The light field camera: Extended depth of field, aliasing, and superresolution
      \item[journal] Pattern Analysis and Machine Intelligence, IEEE Transactions on
      \item[volume] 34
      \item[number] 5
      \item[pages] 972--986
      \item[year] 2012
      \item[publisher] IEEE
    \end{description}
    cite e verifique o resultado.
\item Consulte o \emph{ieeexplore} e inclue mais referências de diversos gêneros. Cite e analise os resultados;
\end{enumerate}

\item Num outro documento (reproduza os pontos \ref{one:one}-\ref{one:five}, da questão 1) reproduza em \LaTeX{} a Tabela \ref{tab:tab}:

\begin{table}[!h]
\centering
\caption{Population between the ages of 15 and 64 as a percentage of the total population.} 
\begin{tabular}{c||c|c|c|c|c}
\hline\hline
World Region             & Country        & 2008    & 2009  & 2010  & 2011\\ \hline\hline
\multirow{2}{*}{Africa}  & Angola         & 50 	    & 51    & 51    & 51\\
                         & South Africa   & 65      & 65    & 65    & 65 \\\hline
\multirow{2}{*}{Asia}    & Afghanistan    & 51      & 51    & 51    & 52\\
                         & China 	  & 72      & 72    & 72    & 73\\\hline
\multirow{2}{*}{South America}   & Argentina     & 64 	& 64 	& 65 	& 65 \\	
                                 & Brazil 	& 67 	& 67 	& 68 	&68 \\
\hline\hline
\end{tabular}
\label{tab:tab}
\end{table}

\item Efetue as seguintes operações:
\begin{itemize}
\item Crie um novo documento .tex (por exemplo: ex3.tex);
\item Neste ficheiro, introduza as instruções para criar um documento \LaTeX{} minimalista;
\item Crie um parágrafo para cujo texto tenha que estar centrado (ver exemplo)
\begin{center}
Isto é um texto centrado.
\end{center}
\item Introduza um parágrafo cujo texto tenha que ficar em \emph{itálico};
\item Introduza um parágrafo com alguns elementos a \textbf{bold};
\item Introduza um parágrafo com \large palavras em \scriptsize diferentes \huge tamanhos;\normalsize
\item Crie texto que esteja repartido\\por duas linhas;
\item Crie um parágrafo em \texttt{true type};
\item Crie um parágrafo do Lorem Ipsum em verbatim \verb!\lipsum[1]!
\item Crie uma nota de rodapé\footnote{que se assemelhe a este exemplo.};	
\item Intruduza um \verb!url! no seu documento;
\item Experimente incluir no preâmbulo o módulo hyperref;
\item No preâmbulo, a seguir à inclusão do módulo hyperref, introduza as seguintes instruções:\\
\begin{verbatim}
\hypersetup{
colorlinks=true,
linkcolor=red
}
\end{verbatim}
Analise as diferenças;
\item Consulte a documentação do hyperref e efetue alterações ao seu gosto;
\end{itemize}

\item Crie em \LaTeX{} as seguintes fórmulas (tome atenção à formatação, numeração, $\ldots$):
\begin{enumerate}
\item $x = \frac{-b \pm \sqrt{b^2 - 4ac}}{2a}$

\item 
\[
\sum (1+x)^n
\]

\item
\begin{equation}
e^x=1+\frac{x}{1!}+\frac{x^2}{2!}+\frac{x^3}{3!}+...
\end{equation}

\item 
\begin{equation}
f(x)=a_0+\sum_{n=1}^\infty \left [ a_n \cos\left ( \frac{nx\,\pi}{L} \right ) + b_n \sin\left ( \frac{nx\,\pi}{L} \right ) \right ]
\end{equation}

\item 
\[
\begin{pmatrix}
 a_{11} & \cdots & a_{1n}\\ 
 \vdots & \ddots & \vdots\\ 
 a_{m1} & \cdots & a_{mn}
 \end{pmatrix}
\]

\item Crie outras fórmulas ao seu gosto. Pesquise o guia do amsmath em \url{ctan.org} para as diversas opções;

\item Crie um sistema de equações como o a seguir apresentado:
\begin{equation}
\hat{y} = \left \{
\begin{array}{ll}
y &= mx + b\\
x &= 3
\end{array}
\right .
\end{equation}

\end{enumerate}

\end{enumerate}

